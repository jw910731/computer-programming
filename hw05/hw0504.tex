 \documentclass[a4paper,10pt]{article}

\usepackage{listings}
\usepackage{xltxtra}
\usepackage{fontspec} %設定字體
\usepackage[CheckSingle, CJKmath]{xeCJK}
\usepackage{color}
\usepackage[x11names]{xcolor}
\usepackage{csquotes}

\usepackage{hyperref}
\hypersetup{
    colorlinks=true,
    linkcolor=blue,
    filecolor=magenta,
    urlcolor=blue,
}

\setCJKmainfont{Noto Serif CJK TC}
\setromanfont[Mapping=tex-text]{Latin Modern Roman}
\setsansfont[Mapping=tex-text]{Latin Modern Sans}
\setmonofont[Mapping=tex-text]{Fira Code}

\lstset{										% Code顯示
    basicstyle=\footnotesize\ttfamily, 					% the size of the fonts that are used for the code
    numbers=left,									% where to put the line-numbers
    numberstyle=\footnotesize,					% the size of the fonts that are used for the line-numbers
    stepnumber=1,									% the step between two line-numbers. If it's 1, each line  will be numbered
    numbersep=5pt,									% how far the line-numbers are from the code
    backgroundcolor=\color{white},				% choose the background color. You must add \usepackage{color}
    showspaces=false,								% show spaces adding particular underscores
    showstringspaces=false,						% underline spaces within strings
    showtabs=false,								% show tabs within strings adding particular underscores
    frame=false,										% adds a frame around the code
    tabsize=4,										% sets default tabsize to 2 spaces
    captionpos=b,									% sets the caption-position to bottom
    breaklines=true,								% sets automatic line breaking
    breakatwhitespace=false,						% sets if automatic breaks should only happen at whitespace
    escapeinside={\%*}{*)},						% if you want to add a comment within your code
    morekeywords={*},								% if you want to add more keywords to the set
    keywordstyle=\bfseries\color{Blue1},
    commentstyle=\itshape\color{Red4},
    stringstyle=\itshape\color{Green4},
}

%================================================%

\title{C Programming II\\HW0504\\Writing Assignment Answer}
\author{作者: 吳文元(jw910731)}
\date{日期: 2021/06/04}

\begin{document}
\maketitle
講到這兩個 operator ,就必須說明 token 。 Token 粗略的可以解釋為程式碼中,帳面上的字,而不是裡面所代表的意思。

比如說我們在程式中寫\lstinline{int a = 10;},此時\lstinline{a}裡面所代表的意思是他的記憶體位置,而他的token就是a,字面上的a。
\section{\# operator}
\begin{lstlisting}[language=c]
#include <stdio.h>
#define mkstr(s) #s
int main(void){       
    printf(mkstr(geeksforgeeks));
    return 0;
}
\end{lstlisting}
這支程式中的巨集\lstinline{mkstr}會將傳進去的 token 換成字串。
所以這隻程式會輸出\lstinline{geeksforgeeks}。
\subsection{\#的範例}
\begin{lstlisting}[language=c, caption={\#範例程式}]
#include <stdio.h>
#define CLR_24_FG(R,G,B) ASCII_ESC"[38;2;"#R ";"#G ";"#B "m"
#define CLR_24_BG(R,G,B) ASCII_ESC"[48;2;"#R ";"#G ";"#B "m"
int main() {
    printf(CLR_24_FG(0, 0, 255) "hello\n"CLR_RST);
    return 0;
}
\end{lstlisting}
這個範例中,我用\#來將輸入進去的 token 直接換成字串,方便我設定 24 色的終端機色彩文字
\section{\#\# operator}
\begin{lstlisting}[language=c, escapeinside={(*}{*)}]
#include <stdio.h>
#define concat(a, b) a##b
int main(void){
    int xy = 30;
    printf("%d", concat(x, y)); (*\label{code:concat}*)
    return 0;
}
\end{lstlisting}
\#\#運算子會將兩個 token 合在一起,這隻程式的第\ref{code:concat}行會被替換成\lstinline{printf("%d", xy);},因此會輸出30,也就是\lstinline{xy}的值。

\subsection{\#\#範例}
\begin{lstlisting}[language=c, caption={\#\#範例程式\protect\footnotemark}]
#include <stdio.h>

#define CAT_INTERNAL(a, b) a##b
#define CAT(a, b) CAT_INTERNAL(a, b)

#define LAMBDA(return_type, fn_body) \
    ({ return_type CAT(lambda_, __LINE__) fn_body CAT(lambda_, __LINE__); })

int main() {
    void (*lambda)(int) = LAMBDA(
        void, (int a) { printf("hello: %d\n", a); });
    lambda(10);
    return 0;
}
\end{lstlisting}
\footnotetext{這支程式不符合C語言標準,請使用GCC編譯器並在編譯時加上參數\lstinline{--std=gnu11}}

這支範例是用來實現「匿名函數」的功能,這個 macro 可以原地製造出一個 function pointer,其中為了避免命名衝突,我的函數會命名為\lstinline{lambda_<Line Number>};為了要達到件事,我用到了\#\#運算子來達成。

其中為了避免\lstinline{__LINE__}巨集因為合併造成沒辦法辨認成行號,因此必須編寫功能性的巨集來包裝\#\#運算子。
\end{document}
