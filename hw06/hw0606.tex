\documentclass[a4paper,10pt]{article}

\usepackage{listings}
\usepackage{xltxtra}
\usepackage{fontspec} %設定字體
\usepackage[CheckSingle, CJKmath]{xeCJK}
\usepackage{color}
\usepackage[x11names]{xcolor}

\setCJKmainfont{Noto Sans CJK TC}
\setromanfont[Mapping=tex-text]{Latin Modern Roman}
\setsansfont[Mapping=tex-text]{Latin Modern Sans}
\setmonofont[Mapping=tex-text]{Latin Modern Mono}

\lstset{										% Code顯示
    basicstyle=\footnotesize\ttfamily, 					% the size of the fonts that are used for the code
    numbers=left,									% where to put the line-numbers
    numberstyle=\footnotesize,					% the size of the fonts that are used for the line-numbers
    stepnumber=1,									% the step between two line-numbers. If it's 1, each line  will be numbered
    numbersep=5pt,									% how far the line-numbers are from the code
    backgroundcolor=\color{white},				% choose the background color. You must add \usepackage{color}
    showspaces=false,								% show spaces adding particular underscores
    showstringspaces=false,						% underline spaces within strings
    showtabs=false,								% show tabs within strings adding particular underscores
    frame=false,										% adds a frame around the code
    tabsize=2,										% sets default tabsize to 2 spaces
    captionpos=b,									% sets the caption-position to bottom
    breaklines=true,								% sets automatic line breaking
    breakatwhitespace=false,						% sets if automatic breaks should only happen at whitespace
    escapeinside={\%*}{*)},						% if you want to add a comment within your code
    morekeywords={*},								% if you want to add more keywords to the set
    keywordstyle=\bfseries\color{Blue1},
    commentstyle=\itshape\color{Red4},
    stringstyle=\itshape\color{Green4},
}

%================================================%

\title{C Programming I\\HW0606\\Answer}
\author{作者: 吳文元(jw910731)}
\date{日期: 2021/01/10}

\begin{document}
\maketitle
\section{Bonus: What is the difference}
兩支程式的執行結果完全相同,但其中記憶體位置的分佈卻有所不同
我將兩隻程式皆加入以下程式片段在主程式的最後面
\begin{lstlisting}[language=c]
for (size_t i = 0; i < 9; i++){
    printf("%p\n", a[i]);
}
printf("%p\n", a);
\end{lstlisting}
\subsection{程式1}
這是加入的程式之輸出 :
\begin{lstlisting}
0x7ffddbe784a0
0x7ffddbe784c4
0x7ffddbe784e8
0x7ffddbe7850c
0x7ffddbe78530
0x7ffddbe78554
0x7ffddbe78578
0x7ffddbe7859c
0x7ffddbe785c0
0x7ffddbe784a0
\end{lstlisting}
我們可以發現每行記憶體都相差9*4個Byte,其中的4是\lstinline{int_32}的大小。
也就是說每一行的位置是完全相鄰、連續的,且完全由Stack管理(也就是自動管理其記憶體的生命周期)。
\subsection{程式2}
這是加入的程式之輸出:
\begin{lstlisting}
0x556326a1f2a0
0x556326a1f2d0
0x556326a1f300
0x556326a1f330
0x556326a1f360
0x556326a1f390
0x556326a1f3c0
0x556326a1f3f0
0x556326a1f420
0x7ffc974f07b0
\end{lstlisting}
雖然可以發現每行記憶體相差12*4個Byte,但這其實只是實作導致的巧合,此外陣列a的第一個維度與其他維度所在的位置是相差甚遠的!
這是因為a這個陣列是在Stack上,而其元素指向的位置皆在Heap上,且由\lstinline{malloc}與\lstinline{free}函數手動管理這些元素指向的空間的生命周期!
此外,這支程式有記憶體洩漏的隱憂存在,因其\lstinline{malloc}的空間沒有被\lstinline{free}掉
\subsection{結論}
這也就是說這兩支程式最大的差別是在記憶體管理的策略上,第1支程式使用自動的管理策略,將陣列放在Stack上面;第2支程式使用手動的管理策略,宣告了一個指標陣列在Stack上,且裡面的元素指向由\lstinline{malloc}分配的Heap空間,但顯然管理不當,出現了記憶體洩漏的問題,幸好這支程式很快就執行結束了,並未造成問題。
\end{document}
