\documentclass[a4paper,10pt]{article}

\usepackage{listings}
\usepackage{xltxtra}
\usepackage{fontspec} %設定字體
\usepackage[CheckSingle, CJKmath]{xeCJK}
\usepackage{color}
\usepackage[x11names]{xcolor}

\usepackage{hyperref}
\hypersetup{
    colorlinks=true,
    linkcolor=blue,
    filecolor=magenta,      
    urlcolor=blue,
}

\setCJKmainfont{Noto Serif CJK TC}
\setromanfont[Mapping=tex-text]{Latin Modern Roman}
\setsansfont[Mapping=tex-text]{Latin Modern Sans}
\setmonofont[Mapping=tex-text]{Fira Code}

\lstset{										% Code顯示
    basicstyle=\footnotesize\ttfamily, 					% the size of the fonts that are used for the code
    numbers=left,									% where to put the line-numbers
    numberstyle=\footnotesize,					% the size of the fonts that are used for the line-numbers
    stepnumber=1,									% the step between two line-numbers. If it's 1, each line  will be numbered
    numbersep=5pt,									% how far the line-numbers are from the code
    backgroundcolor=\color{white},				% choose the background color. You must add \usepackage{color}
    showspaces=false,								% show spaces adding particular underscores
    showstringspaces=false,						% underline spaces within strings
    showtabs=false,								% show tabs within strings adding particular underscores
    frame=false,										% adds a frame around the code
    tabsize=4,										% sets default tabsize to 2 spaces
    captionpos=b,									% sets the caption-position to bottom
    breaklines=true,								% sets automatic line breaking
    breakatwhitespace=false,						% sets if automatic breaks should only happen at whitespace
    escapeinside={\%*}{*)},						% if you want to add a comment within your code
    morekeywords={*},								% if you want to add more keywords to the set
    keywordstyle=\bfseries\color{Blue1},
    commentstyle=\itshape\color{Red4},
    stringstyle=\itshape\color{Green4},
}

%================================================%

\title{C Programming I\\HW0106\\Answer}
\author{作者: 吳文元(jw910731)}
\date{日期: 2021/03/15}

\begin{document}
\maketitle
\section{Bonus: perror}

\lstinline{perror}的功能就是可以將\lstinline{errno}對應的錯誤訊息打印到\lstinline{stderr}去,使用時可以傳一個字串作為輸出時的前綴,方便編寫者追蹤錯誤的原因!

其實說穿了,\lstinline{perror}可以被視為\lstinline{strerror}的包裝,透過封裝\lstinline{strerror}並將他的輸出與給予的參數的一併輸出到\lstinline{stderr}的輸出流上!兩者的差異就是,我如果使用\lstinline{strerror},我可以自己選擇要輸出到哪個輸出流,甚至是儲存錯誤資訊,提供更大的彈性,而\lstinline{perror}則是較為方便,將常用的錯誤資訊處理方式封裝起來,使除錯更方便!

\subsection{\lstinline{perror}範例}
這裡提供我之前寫的簡易沙盒作為\lstinline{perror}的範例,完整的專案請見\href{https://github.com/jw910731/simple-sandbox}{simple-sandbox}

\begin{lstlisting}[language=c++, label={lst:perror Example}, caption={perror Example in \lstinline{src/sandbox.cpp}}, firstnumber=136]
void Sandbox::child(const std::vector<std::string> &args) {
    // prepare args and env
    char *const *prepared_args = prepare_helper(args);
    // preserve for env passing
    char *const *prepared_envs = prepare_helper({});
    // setup fd for redirection
    setupFd();
    // setup rlimit
    setupLimit();
    // execute real program
    int ret = execve(filePath.c_str(), prepared_args, prepared_envs);
    if(ret < 0){
        perror("execve():");
        exit(-1);
    }
}
\end{lstlisting}

可以看到裡面使用了\lstinline{perror}來捕捉\lstinline{execve}所遇到的錯誤,供使用者與程式設計師除錯,在呼叫時,傳入了參數\lstinline{execve():}使得在印出錯誤訊息時,可以知道是哪一個函數呼叫失敗並可以得知失敗的原因!
這樣如果使用者所給定的執行檔不存在或不具有可執行權限,便可在錯誤訊息中得知!
\end{document}