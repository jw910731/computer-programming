\documentclass[a4paper,10pt]{article}

\usepackage{listings}
\usepackage{xltxtra}
\usepackage{fontspec} %設定字體
\usepackage[CheckSingle, CJKmath]{xeCJK}
\usepackage{color}
\usepackage[x11names]{xcolor}
\usepackage{csquotes}

\usepackage{hyperref}
\hypersetup{
    colorlinks=true,
    linkcolor=blue,
    filecolor=magenta,
    urlcolor=blue,
}

\setCJKmainfont{Noto Serif CJK TC}
\setromanfont[Mapping=tex-text]{Latin Modern Roman}
\setsansfont[Mapping=tex-text]{Latin Modern Sans}
\setmonofont[Mapping=tex-text]{Fira Code}

\lstset{										% Code顯示
    basicstyle=\footnotesize\ttfamily, 					% the size of the fonts that are used for the code
    numbers=left,									% where to put the line-numbers
    numberstyle=\footnotesize,					% the size of the fonts that are used for the line-numbers
    stepnumber=1,									% the step between two line-numbers. If it's 1, each line  will be numbered
    numbersep=5pt,									% how far the line-numbers are from the code
    backgroundcolor=\color{white},				% choose the background color. You must add \usepackage{color}
    showspaces=false,								% show spaces adding particular underscores
    showstringspaces=false,						% underline spaces within strings
    showtabs=false,								% show tabs within strings adding particular underscores
    frame=false,										% adds a frame around the code
    tabsize=4,										% sets default tabsize to 2 spaces
    captionpos=b,									% sets the caption-position to bottom
    breaklines=true,								% sets automatic line breaking
    breakatwhitespace=false,						% sets if automatic breaks should only happen at whitespace
    escapeinside={\%*}{*)},						% if you want to add a comment within your code
    morekeywords={*},								% if you want to add more keywords to the set
    keywordstyle=\bfseries\color{Blue1},
    commentstyle=\itshape\color{Red4},
    stringstyle=\itshape\color{Green4},
}

%================================================%

\title{C Programming I\\HW0205\\Answer}
\author{作者: 吳文元(jw910731)}
\date{日期: 2021/03/20}

\begin{document}
\maketitle
\begin{lstlisting}[language=c, caption={題敘中的程式}, escapeinside={(*}{*)}]
#include <stdio.h>
#include <stdint.h>

int main()
{
    int32_t number = 0;

    scanf( "%d", & number );

    int32_t bit = 1;
    bit = bit << 31; (*\label{code:bug}*)
    
    for( int i = 0 ; i < 32 ; i++ ){
        if( bit & number )
        printf( "1" );
        else
        printf( "0" );
        bit = bit >> 1;
    }
    return 0;
}
\end{lstlisting}
其中第\ref{code:bug}行中的左移在規格書中是未定義行為!
根據規格書\href{https://drive.google.com/file/d/1BGSbEazY5azERlFtP9DfmTTHgj-0AjY7/view}{ISO/IEC 9899:201x}中的第95頁指出
\begin{displayquote}
The result of E1 << E2 is E1 left-shifted E2 bit positions; vacated bits are filled with
zeros. If E1 has an unsigned type, the value of the result is \(E1 × 2^{E2}\), reduced modulo
one more than the maximum value representable in the result type. If E1 has a signed
type and nonnegative value, and \(E1 × 2^{E2}\) is representable in the result type, then that is
the resulting value; otherwise, the behavior is undefined.
\end{displayquote}
而我們可以發現程式中第\ref{code:bug}行,裡面表達的\lstinline{1<<31}所指的\(1\times 2^{31}\)已超出32位有號整數所能儲存的\([2^{31}-1, -2^{31}]\)範圍了,
使得這個expression的行為將會是platform specific的。

這隻程式應該使用\lstinline{union},將有號整數與無號整數共用同個空間,並使用無號整數做安全的位元運算!我修改後如下:

\begin{lstlisting}[language=c, caption={修改後的的程式}]
#include <stdio.h>
#include <stdint.h>

union soviet{
    int32_t num;
    uint32_t usgn_num;
};

int main()
{
    union soviet number;

    scanf( "%d", & number.num );

    uint32_t bit = 1;
    bit = bit << 31;
    
    for( int i = 0 ; i < 32 ; i++ ){
        if( bit & number.usgn_num )
        printf( "1" );
        else
        printf( "0" );
        bit = bit >> 1;
    }
    return 0;
}
\end{lstlisting}
這樣就可以避免掉不安全且不可移植的行為了!
\end{document}
